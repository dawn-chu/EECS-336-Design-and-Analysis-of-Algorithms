%%%%%%%%%%%%%%%%%%%%%%%%%%%%%%%%%%%%%%%%%
% Short Sectioned Assignment
% LaTeX Template
% Version 1.0 (5/5/12)
%
% This template has been downloaded from:
% http://www.LaTeXTemplates.com
%
% Original author:
% Frits Wenneker (http://www.howtotex.com)
%
% License:
% CC BY-NC-SA 3.0 (http://creativecommons.org/licenses/by-nc-sa/3.0/)
%
%%%%%%%%%%%%%%%%%%%%%%%%%%%%%%%%%%%%%%%%%

%----------------------------------------------------------------------------------------
%	PACKAGES AND OTHER DOCUMENT CONFIGURATIONS
%----------------------------------------------------------------------------------------

\documentclass[paper=a4, fontsize=11pt]{scrartcl} % A4 paper and 11pt font size

\usepackage[T1]{fontenc} % Use 8-bit encoding that has 256 glyphs
\usepackage{fourier} % Use the Adobe Utopia font for the document - comment this line to return to the LaTeX default
\usepackage[english]{babel} % English language/hyphenation
\usepackage{amsmath,amsfonts,amsthm} % Math packages

\usepackage{lipsum} % Used for inserting dummy 'Lorem ipsum' text into the template

\usepackage{indentfirst}
\setlength{\parindent}{2em}

\usepackage{sectsty} % Allows customizing section commands
\allsectionsfont{\centering \normalfont\scshape} % Make all sections centered, the default font and small caps

\usepackage{fancyhdr} % Custom headers and footers
\pagestyle{fancyplain} % Makes all pages in the document conform to the custom headers and footers
\fancyhead{} % No page header - if you want one, create it in the same way as the footers below
\fancyfoot[L]{} % Empty left footer
\fancyfoot[C]{} % Empty center footer
\fancyfoot[R]{\thepage} % Page numbering for right footer
\renewcommand{\headrulewidth}{0pt} % Remove header underlines
\renewcommand{\footrulewidth}{0pt} % Remove footer underlines
\setlength{\headheight}{13.6pt} % Customize the height of the header

\numberwithin{equation}{section} % Number equations within sections (i.e. 1.1, 1.2, 2.1, 2.2 instead of 1, 2, 3, 4)
\numberwithin{figure}{section} % Number figures within sections (i.e. 1.1, 1.2, 2.1, 2.2 instead of 1, 2, 3, 4)
\numberwithin{table}{section} % Number tables within sections (i.e. 1.1, 1.2, 2.1, 2.2 instead of 1, 2, 3, 4)

\setlength\parindent{0pt} % Removes all indentation from paragraphs - comment this line for an assignment with lots of text

%----------------------------------------------------------------------------------------
%	TITLE SECTION
%----------------------------------------------------------------------------------------

\newcommand{\horrule}[1]{\rule{\linewidth}{#1}} % Create horizontal rule command with 1 argument of height

\title{	
\normalfont \normalsize 
\textsc{Northwestern University} \\ [25pt] % Your university, school and/or department name(s)
\horrule{0.5pt} \\[0.4cm] % Thin top horizontal rule
\huge EECS 336 -- Homework 1 \\ % The assignment title
\horrule{2pt} \\[0.5cm] % Thick bottom horizontal rule
}

\author{Weihan Chu} % Your name

\date{\normalsize\today} % Today's date or a custom date

\begin{document}

\maketitle % Print the title

%----------------------------------------------------------------------------------------
%	PROBLEM 1
%----------------------------------------------------------------------------------------

\section{Question 1}

\subsection{Question 1 Part a}
From the question of part a, we need to order following 12 functions by growth:

\begin{align*}
(\frac{3}{2})^{2n},
n^3,
(lg^2n)^2,
lg(n!),
\left(2^{2^n}\right)^2,
n^{\frac{1}{lgn}}\\ 
\left(ln(lnn)\right)^2,
lg^*n,
\left(n 2^n\right)^2,
n^{lglgn},
\left(lnn\right)^2,
1\\
\end{align*}

ranking can be solved by the following rules:\\
1.exponential functions grow faster than polynomial functions than logarithmic functions\\
2.the base of logarithm does not matter \\
3.$n^{lglgn}=({lgn})^{lgn}$\\
4.$lg(n!)=\Theta(nlgn)$\\
5.$n^{\frac{1}{lgn}}=2$\\ 

So the final order will be:
\begin{align*}
\left(2^{2^n}\right)^2>\left(n 2^n\right)^2>(\frac{3}{2})^{2n}>n^{lglgn}>n^3>lg(n!)>(lg^2n)^2>\left(lnn\right)^2>\left(ln(lnn)\right)^2>lg^*n> n^{\frac{1}{lgn}} and 1
\end{align*}
%------------------------------------------------

\subsection{Question 1 Part b}
As said on the piazza,we can find one upper bound function and one lower bound function for all 12 functions and then modify those two functions to become strict upper bound function and strict lower bound function. Then make a combination.

So the final result I get is:

\begin{equation*}
\begin{split}
f(n)=\left({(1+cosx)} 2^{2^n}\right)^2
\end{split}					
\end{equation*}
%----------------------------------------------------------------------------------------
%	PROBLEM 2
%----------------------------------------------------------------------------------------
\vspace{2cm}
\section{Question 2}
\subsection{Question 2 Part a}
\begin{align*}
\begin{split}
f(n) &=lg(lg^*n)+2^{lgn}*{lgn}^{lgn}\\
 &=\Theta(1)+n^{lg(lgn)+1}\\
 &=\Theta(n^{lg(lgn)+1})
\end{split}
\end{align*}

\subsection{Question 2 Part b}
\begin{align*}
\begin{split}
f(n) &=2^{lg^* n}*log({lgn}^{lgn})+{4^{lgn}}^3\\
&=\Theta(1)+lgn*lg(lgn)+2^{6lgn}\\
&=\Theta(1)+lgn*lg(lgn)+\Theta(n^6)\\
&=\Theta(n^6)
\end{split}
\end{align*}

\subsection{Question 2 Part c}
\begin{align*}
\begin{split}
f(n) &={(\sqrt 2)}^{lgn}+e^n+log(\sqrt {lgn})\\
&=\sqrt n+\Theta(e^n)+\Theta(lg(lgn))\\
&=\Theta(e^n)
\end{split}
\end{align*}

\subsection{Question 2 Part d}
\begin{align*}
\begin{split}
f(n) &=n^2*{4^{lgn}}^3+(\sqrt 2)^{lgn}\\
&=\Theta(n^8)+(\sqrt 2)^{lgn}\\
&=\Theta(n^8)
\end{split}
\end{align*}

\subsection{Question 2 Part e}
\begin{align*}
\begin{split}
f(n) &=n!*(lgn)!*(\sqrt lgn)+2^{lg^* n}\\
&=n!*(lgn)!*(\sqrt lgn)+\Theta(1)\\
&=n^{n+\frac{1}{2}}*e^{-n}*{lgn}^{lgn}*(lgn)\\
&=\Theta(lgn*n^{lg(lgn)+n+\frac{1}{2}}*e^{-n})
\end{split}
\end{align*}

\subsection{Question 2 Part f}
\begin{align*}
\begin{split}
f(n)= &=2^{(n+1)!}+{(\sqrt 2)}^{lgn}\\
&=2^{(n+1)!}+\Theta(n)\\
&=\Theta(2^{n^{n+\frac{1}{2}}*e^{-n}})
\end{split}
\end{align*}
%----------------------------------------------------------------------------------------
%	PROBLEM 3
%----------------------------------------------------------------------------------------
\vspace{2cm}
\section{Question 3}
\subsection{Question 3 Part a}
\mbox{if the condition is $f(n)\notin \overset{\infty}{\Omega}(g(n))$}, \mbox{so there is only finite integer which satisfies $f(n)>=cg(n)$ },
\mbox{if $n_0$ is the maximal value of all $n$ },
\mbox{which means for the $n>={n_0}+1$,$cg(n)>=f(n)>=0$}.\\
\mbox{so it will satisfies either of the following condition of $f(n)\in \overset{\infty}{\Omega}(g(n))$ or $f(n)\in O(g(n))$}.

\subsection{Question 3 Part b}
Advantage:it can analyze a large more range of complexity, especially when Omega is hard to find.
Disadvantage: The range is too big,which will influence the accuracy and the range is not fixed.

\subsection{Question 3 Part c}
\mbox{if we have$f(n)=O^{'}(g(n))$, if and only if $|f(n)|=O(g(n))$, }\\
\mbox{from $f(n)=\Theta(g(n))$,}\\
\mbox{we can get $|f(n)|=O(g(n))$ and $f(n)=\Omega(g(n))$,}\\
\mbox{then from $|f(n)|=O(g(n))$ and $f(n)=\Omega(g(n))$,}\\
\mbox{we can't get $f(n)=\Theta(g(n))$ }\\
So the direction change from two-way to one direction

\subsection{Question 3 Part d}
$ \overset{\sim}{\Omega}=f(n) $\\
$ f(n)\geq cg(n)lg^k (n) $  for all $n \geq n_0 $ \\
$ \overset{\sim}{\Theta}=f(n) $\\
$ cg(n)lg ^k(n)\leq f(n)\leq cg(n)lg^k (n) $ for all $n \geq n_0 $ \\


\vspace{2cm}
\section{Question 4}
\subsection{Question 4 Part a}
$f(n)=n-1,c=2$ \\
then: $n-k\leq2, k\geq n-2$ \\
and: $ n-(k-1)>2,k<n-3$\\
so: $ k=\Theta(n)$

\subsection{Question 4 Part c}
$f(n)=\frac{n}{2},c=3 $\\
then: ${(\frac{n}{2})}^{k} \leq 3 , k \geq lg(\frac{n}{3})$\\
and: ${(\frac{n}{2})}^{k-1} > 3 , k < lg(\frac{n}{3})-1$\\
so: $ k=\Theta lg(n) $

\subsection{Question 4 Part e}
$f(n)=\sqrt{n},c=4 $\\
then: $ (n)^{\frac{1}{2^k}} \leq 4,  k\geq lg(lgn)-1 $\\
and: $ (n)^{\frac{1}{2^(k-1)}} > 4, k< lg(lgn) $\\
so: $ k= \Theta lg(lg n) $

\subsection{Question 4 Part g}
$f(n)=n^{\frac{1}{3}},c=4$ \\
then $ (n)^{\frac{1}{3^k}} \leq 4, k\geq \frac{lg(lgn)-1}{lg3}$\\
and: $ (n)^{\frac{1}{3^{k-1}}} > 4, k< \frac{lg3-1+lg(lgn)}{lg3}$\\
so: $ k= \Theta lg(lg n) $

\subsection{Question 4 Part h}
$f(n)=\frac{n}{lgn},c=4$\\
then: $ \frac{n}{\sqrt n} \leq \frac{n}{lgn} \leq \frac{n}{2} $\\
then for the left: $f(n)=\frac{n}{\sqrt n}= \sqrt{n}, c=4  $\\
and for the left: $ (n)^{\frac{1}{2^{k_1}}} \leq 4,  k_1\geq lg(lgn)-1 $\\
and for the left: $ (n)^{\frac{1}{2^(k_1-1)}} > 4, k_1< lg(lgn) $\\
so for the left: $ k_1= \Theta lg(lg n) $ \\
so for the left: $k_1=\Omega(lg(lgn))$\\
then for the right:
$f(n)=\frac{n}{2},c=4$\\
then for the right: ${(\frac{n}{2})}^{k_2} \leq 3 , k_2 \geq lg(\frac{n}{3})$\\
and for the right: ${(\frac{n}{2})}^{k_2-1} > 3 , k < lg(\frac{n}{3})-1$\\
so for the right: $ k_2=\Theta lg(n) $\\
so for the right:$ k_2=O(lgn)$\\
so: then lower bound is :$lg(lgn) $\\
the upper bound is :$ lgn $
\end{document}